The interplay of topology, quantum information, and condensed matter physics has emerged as  of the most fascinating and quickly developing areas of modern . 
In Philip Anderson's famous 1972 article ``More is Different,'' he describes how the collective behavior of large complex aggregates can drastically differ from that of the constituents~\cite{Anderson72}. 
Typically these macroscopic behaviors are distinguished by local symmetry breaking order parameters.
A famous counterexample came with the discovery of the quantum Hall effect, where distinct phases enjoyed the same symmetries and thus could not be fully characterized by local order parameters al.
This discovery marked the introduction of topological phases of matter and triggered decades of  on the subject.
Topological phases provide an organizing frame for a wide variety of problems from the quantum Hall effect to error-correcting codes.

Inherently robust to local perturbations, topological phases furnish platforms well-suited for fault-tolerant quantum computation.
Quantum computers are  of the major driving forces of  on topological phases today and promise a wide range of applications from testing modern security protocols to quantum simulators that can predict the physics of real systems beyond the capabilities of any classical computer.
Realizing this potential is burdened by many fundamental challenges such as environmental noise.
 resolution exploited in nearly all quantum computing platforms is the  of topology---at its core,  delocalizes quantum information over space, ensuring that it is insensitive to local perturbations.



{
Inspired by the  for realistic blueprints of quantum computing hardware, my  spans several areas of topological quantum condensed matter physics. On the practical side, I propose realizations of topological phases and highlight their applications toward quantum computing technologies~\cite{Aasen2016,Aasen20,Slagle2019,Klocke2020}. And on the more formal end, I characterize and classify topological phases by developing the fundamental mathematics underpinning their description~\cite{ALM,Aasen2020SPT,Aasen2020defect}. 
The active  community at the Stewart Blusson Quantum Matter Institute complements my  program and has enough overlap to allow for creative interdisciplinary collaboration across the gamut of theorists and experimentalists.
Growing up in Vancouver situates me in an ideal position to lead outreach initiatives within the community and develop industry-based  collaborations with local companies. For me, the Stewart Blusson Quantum Matter Institute offers a unique opportunity, which cannot be matched by any other institution in the world, making it a dream position.
This  statement outlines several short and long-term  directions I would pursue in my group. 
}


{Towards quantum computation with topological materials}

A gapped -dimensional phase of matter whose low energy excitations are non-Abelian anyons can be  to build fault-tolerant topological qubits. Associated with multiple well-separated non-Abelian anyons is an expntially large ground state degeneracy that can be  to encode quantum information. Crucially, this quantum information is inaccessible to any local measurement or perturbation and hence protected from environmental noise. Adiabatic manipulation of the anyons, such as braiding, performs rigid non-commutative unitary rotations on the  of degenerate ground states;  Fig.~\ref{SLfig}(a).
\begin{figure}[htbp]
   \centering
   \includegraphics[width=.99\columnwidth]{SLFigv2.pdf}
   \caption{(a) Braiding process of  non-Abelian anyons leading to the transformation $|\Psi \rangle \mapsto U | \Psi \rangle$. 
{   (b) A topological qubit assembled from weakly correlated materials allowing for fusion rule detection.}
   (c) A fermion condensate stabilized by a superconductor allows physical fermions at the edge of a $\nu=1$ quantum Hall state to be converted into emergent fermions at the edge of a non-Abelian spin liquid.
{   The bulk anyon $x$ can be detected by a universal interferometric signal in the current from the source to the drain.}
   (d) Cartoon of a trapping potential determined from the magnetization change $\Delta M$.
   }
   \label{SLfig}
\end{figure}
The encoded quantum information can be read out by so called fusion, where bringing pairs of anyons are brought together and their shared quantum state is measured. 
During my Ph.D., my collaborators and I introduced a series of ``Milests" which demonstrate the non-Abelian character of Majorana zero modes with next-generation experiments~\cite{Aasen2016}. We  our sights on fusion-rule detection and topological qubit validation;  Fig.~\ref{SLfig}(b).
Extending these blueprints to other topological materials remains a top priority as it broadens the scope of possible topological quantum hardware. Fundamentally, we  a topological material which hosts non-Abelian anyons and is amenable to a high degree of control, allowing for the creation and manipulation of non-Abelian anyons as well as schemes to encode and read out quantum information.

Recent theoretical and experimental advances in materials science have brought to light  candidate non-Abelian topological materials in spin-orbit-coupled Mott insulators~\cite{Jackeli2009}.
In particular, \alpRuCl{3} has gained momentum with recent experiments reporting a thermal Hall conductance consistent with the non-Abelian Ising topological order~\cite{Kasahara2017}.
With initial experiments on non-Abelian spin liquids in place, 
it is natural to ask what steps  to be taken en route to realizing a non-Abelian spin liquid-based-quantum computer.


During my Ph.D., I developed some of the core mathematical formalism ed to classify and characterize {\em intrinsically} fermionic topological phases of matter through a technique called `fermion condensation'~\cite{ALM}. A fermion condensate can form when a composite particle consisting of a low energy physical and emergent fermion condenses;  Fig.~\ref{SLfig}(c). We advanced the mathematical idea of fermion condensation into a physical realization during my postdoc and offered  electrical techniques for probing non-Abelian spin liquids~\cite{Aasen20}.
The fermion condensate allows a physical fermion to be perfectly converted into an emergent fermion at the spin liquid edge. The conversion of physical to emergent fermions enabled us to design circuits that allowed for electrical detection of edge states and interferometric measurements of bulk anyons;  Fig.~\ref{SLfig}(c). This  advances non-Abelian spin liquids toward feasible topological quantum computing hardware, and it opens up many exciting opportunities to explore in future , such as the design of bulk quasiparticle traps, blueprints for  exotic spin liquids, and techniques to stabilize the fermion condensate.



\hfill \break
%\linebreak
%\vspace*{0.1cm}
{\large \textbf{Quasiparticle traps, edge probes, and engineered spin liquids.}}
Before non-Abelian spin liquids can rival conventional topological quantum computing hardware, additional probes must be developed.
An essential compnt will be to determine ways in which we can trap and manipulate non-trivial bulk quasiparticles.
In a spin liquid, bulk quasiparticles do not carry intrinsic quantum numbers, 
making the design of quasiparticle traps inherently difficult. 
This is in contrast to their quantum Hall counterparts, such as the Moore-Read state, where the quasiparticles carry charge and therefore can be trapped in local electric potentials~\cite{Sarma2005,Bonderson2006}. 
I propose to  an extensive symmetry analysis to pin down a correspondence between lattice and effective field theory operators where trapping potentials for the emergent fermion can be more easily established.
Similar techniques could  lead to trapping potentials for the non-Abelian Ising anyon.
For instance, I would first take inspiration from trapping potentials that have been designed in non-Abelian quantum Hall states~\cite{Mong2017}. 
In this approach, we determine the magnetic response of the spin liquid due to inserting a non-Abelian Ising anyon, and then  this magnetic response to determine an ansatz for an external magnetic field that favorably traps a non-Abelian Ising anyon~Fig.\ref{SLfig}(d).
{Once having designed a trapping potential, its occupancy could be detected experimentally using the interferometric techniques we developed in references~\cite{Aasen20} and~\cite{Klocke2020}. A carefully designed array of quasiparticle traps and interferometric measurements would allow non-trivial fusion rule detection experiments. Similar architectures with dynamical traps would implement measurement-based braiding experiments. Such a series of experiments would advance quantum spin liquids toward topological quantum hardware and  even rival the best leading candidates.
}

The edge of a gapped spin liquid contains information about the bulk phase.
Every bulk quasiparticle when brought to the edge leaves a non-trivial and in principle detectable imprint.
Moreover, the edge is well described by a chiral gapless Majorana mode and therefore vulnerable to external perturbations.
When designing circuit elements that made  of fermion condensation, 
we ed at the level of effective field theories.
 drawback of this approach is that it does not provide a microscopic understanding of how to form and stabilize the fermion condensate.
Ultimately, the microscopic interaction necessary to induce the fermion condensate will be a process where an effective spin is formed from a pair of physical fermions that interacts with a physical spin in the spin liquid.
Characterizing how this process is enabled, 
tuned, and stabilized at the microscopic level will be important for future experimental realizations.

{
Engineered quantum systems  provide a potential route to realizing non-Abelian spin liquids. As an example, consider the Kitaev hycomb model---a microscopic realization of the non-Abelian Ising topological order constructed on the hexagonal lattice, with  qubit per site, and  flavors of bond-dependant interactions~\cite{Kitaev2006}. Upon tuning  flavor of bond interaction to zero, the model becomes a  of decoupled Ising chains. Reintroducing these bond interactions provides access to the non-Abelian spin liquid from the much more easily understood Ising chains. Experimental implementation of this construction in a cold atoms system or an array of Josephson junctions should be possible. Naturally operated with strong interactions and a high degree of tunability, these systems make excellent candidates for these so-called coupled chain constructions. The  essential ingredients in realizing the non-Abelian spin liquid are: a critical Ising chain, and appropriate interactions that drive the decoupled chains into the non-Abelian spin liquid. With appropriate interactions, a chain of Rydberg atoms can realize the critical Ising phase~\cite{Fendely2004}, experimental progress in this direction is already well underway~\cite{Bernien2017}. The inter-chain interactions necessary to drive the decoupled chains into a non-Abelian spin liquid have been identified at the field theory level~\cite{Aasen20}. I am actively involved in a  effort to obtain a Rydberg geometry that naturally stabilizes the interactions necessary to realize a non-Abelian spin liquid~\cite{Slagle2020}. Capacitively coupled Josephson junction arrays present another feasible path to an exotic spin liquid. For example, the low energy theory of a linear array of capacitively coupled Josephson junctions is gapless. Such gapless degrees of freedom, along with a high degree of tunability of inter-chain interactions, provide ideal ingredients for destabilizing the chains into exotic 2D phases.
}





%{
%Writing tips: 
%The state of the field is X. 
%But we don't understand Y. 
%Therefore a critical  to do Z. 
 %I propose to do Z building on my previous s U and V.}
{Topological phases in 2+1 dimensions and beyond}
Over the last several decades, topological order has n an impressive evolution from experiment to theory and back.
Topological orders are defined independently from local order parameters, therefore distinct distinct topological phases are indistinguishable by any local measurement.
The topological character of a given phase  present itself in many ways, 
such as quantized edge currents, 
patterns of long-range entanglement,
and exotic topology-dependent ground-state degeneracies.
We can characterize, and sometimes classify, all topological phases of matter under relatively weak assumptions, e.g., the dimension of space, the symmetry group, and the microscopic degrees of freedom.
Developing microscopic realizations of these topological phases is fundamental to the classification, 
in part to understand what degrees of freedom are necessary for experimental realizations, 
and in part to show that the classifications are correct. 
In this section, I propose several  directions related to the classification of bosonic and fermionic topological phases.
I discuss how previous classifications  be related to  another, 
how current classifications  be extended, 
and various techniques for generating  algebraic data  in describing a topological phase.
I  discuss how I would develop exactly solvable models for these novel  phases and how they  guide classifications.


\hfill \break
{\large \textbf{Topological phases with symmetry.}}
A topological phase can be enriched by a global symmetry $G$, e.g., an on-site unitary symmetry such as charge conservation.
The algebraic theory of bosonic topological order with symmetry has been developed in both the mathematical and physics literature under the name of symmetry enriched topological order~\cite{Etingof2009,Bark2019}. 
Symmetry enriched topological orders have point-like symmetry defects, which parameterize the phases' response to insertion of an external symmetry flux.
If the symmetry is non-trivial these defects will harbour non-Abelian anyonic degrees of freedom and can be  to host protected quantum information.
The fact that the defect carries a quantum number determined by the external symmetry flux is highly beneficial, 
as it allows the defects to be manipulated using external degrees of freedom. 



To date, the majority of topological classifications have foc on bosonic topological orders---phases of matter where the microscopic degrees of freedom consist of interacting bosons, such as spins in a Mott insulator.
Much less is known about the classification of their fermionic counterparts---phases of matter constructed from physical fermions.
In a forthcoming ~\cite{Aasen2020SPT} we complete the classification of invertible fermionic phases (i.e., those with a unique ground state on every closed manifold) with on-site symmetry in  spatial dimensions.
In that , we explicitly parameterize the topological data of every $G$-symmetric invertible fermionic phase of matter in  spatial dimensions.
The key insight is that all topological data of the $G$-symmetric invertible fermionic phases can be constructed from the case where $G =\mathbb{Z}_2$, 
which can be solved on a computer or by hand.
This result is extremely surprising, as it collapses the highly non-trivial consistency conditions of {\em all} invertible fermionic phases onto the much simpler  of consistency conditions for $G = \mathbb{Z}_2$.
Previous attempts at classifications have either been ad hoc, as in the defect decoration approach~\cite{Wang2020}, or very technical, as in the spin cobordism classifications~\cite{Kapustin2014}.
Moreover, these previous classification schemes do not produce a complete  of topological data, and instead leave  with an infinite  of consistency conditions to be solved.

The above  approaches to the classification of fermionic phases  vastly different technical tools, yet arrive at the same conclusions.
In particular, I plan to investigate how the $\mathbb{Z}_2$ structure we teased out of the classification appears in the other  classifications.
This structure will likely have equally surprising consequences for both understanding how invertible fermionic orders are constructed using defect decorations,
and the spectral sequences involved in the spin-cobordism classification.
Furthermore, there is good reason to expect the distinguished role played by $\mathbb{Z}_2$ extensions in  spatial dimensions lifts to  spatial dimensions.

\hfill \break
{\large \textbf{Algebraic theory of fermionic phases and toy models.}}
The fundamental theory of fermionic phases of matter is still under development.
An important open question is whether all fermionic phases can be described by condensing a fermion in some bosonic topological order~\cite{ALM}.
Thus far, all known fermionic phases do arise this way, but no formal proof is known.
Such a theorem would drastically simplify the description of fermionic theories---in effect allowing  to simple analyze their bosonic counterparts with minor modifications.
Another line of  is to determine a class of exactly solvable fermionic symmetry enriched topological orders.
In an earlier , I described how to write down a list of fermionic commuting projector models~\cite{ALM}.
It should be straightforward to extend these models to fermionic symmetry enriched topological orders using a similar construction to what is described in~\cite{Heinrich2016,Cheng2017}.
A class of exactly solved models  give key insights into the general structure of fermionic symmetry enriched topological orders. 

\hfill \break
{\large \textbf{Condensation and gauging.}}
To date, there does not exist a systematic construction for producing the universal data of a symmetry enriched topological order.
It is known, however, that in principle, topological boson condensation provides the data necessary to construct a symmetry enriched topological order from a modular tensor category~\cite{Etingof2009}. Elucidating how to construct this data would provide a concrete technique for finding  symmetry enriched topological orders, would tell us when they are proximate to various parent modular tensor categories, and  yield pathways to realizations in solid state systems. This project has both near-term directions, as well as long-term  goals. In the near term, I would first consider condensing a $\mathbb{Z}_2$ boson in a simple class of theories, and determine the corresponding $\mathbb{Z}_2$ symmetry enriched topological order. In the long-term, I would like to extend the $\mathbb{Z}_2$ condensation to more general groups, including non-Abelian s.


The inverse of condensation corresponds to gauging a global symmetry.
Gauging global symmetries in a symmetry enriched topological order is notoriously difficult, and to date no diagrammatic description has been presented.
In particular, there is a related open question: can all fermionic phases be constructed using fermion condensation?
I would approach this question from the side of gauging fermion parity---asking if all bosonic topological orders with an emergent fermion can be found from gauging fermion parity in some fermionic topological phase.

{Universal theory of quantum error correction}
{Nearly all of the currently pursued quantum computing platforms rely on some form of topological protection.
The rapid advancements in superconducting qubits~\cite{Arute2019,IBM2020} and cold atoms systems~\cite{Bernien2017} are heralding the era of scalable quantum computing hardware. Still, designing fault-tolerant qubits from small noisy quantum devices with 50-500 qubits within the next five to ten years present significant challenges. Quantum error-correcting codes will undoubtedly play a vital role in any such effort. Quantum error correcting codes  be viewed through the lens of topological order; the code space corresponds to the degenerate topological ground state subspace, errors to anyonic excitations, and the error correction protocol to detecting and annihilating the anyonic excitations. However, even though topological order admits a concrete universal description in terms of anyon theories~\cite{Kitaev2006}---the same is not true for the topological code counterparts.
}
Developing a universal description that applies beyond intrinsic topological order to topological codes would leverage advances made in the last  decades of topological condensed matter for the quantum information community.
The close relationship between topological codes and topological order makes this a logical, albeit difficult  direction---as such it should be split into several more manageable problems.
For instance, we  to systematically incorporate quantum error correction into the algebraic frame of topological order.
Developing a universal description of topological codes   yield insights into how modular compnts of a quantum computer would be interfaced with  another. 
Below I describe how to make progress in this direction by building off a  of known topological codes and techniques commonly  in studying topological order.
\begin{figure}[htbp]
   \centering
   \includegraphics[width=.650\columnwidth]{code.pdf}
   \caption{Left: a pair of topological orders with alternating boundary conditions. 
   Right: the `merged' topological orders found by performing measurements and error correction along the interface.}
   \label{code}
\end{figure}




Lattice surgery in the toric code provides a convenient entry point for thinking about error correction in topological order~\cite{Horsman2012,Fowler2019}.
Moreover lattice surgery is a current leading proposal for building a scalable quantum computer~\cite{Horsman2012} and is actively being pursued by industry leaders.
The toric code has  distinct gapped boundary conditions characterized by which bulk quasiparticles condense on the boundary.
A single square patch of toric code with alternating boundary conditions can host  logical qubit.
There are  natural fault-tolerant lattice surgery operations which can be d: `merging' and `splitting'. 
The former allows  to take  toric code patches and sew them together by a series of measurements along an interface;  Fig.~\ref{code}. 
The latter allows  to split a single toric code patch into a pair by a series of measurements along a path connecting opposing sides.
The process of merging and splitting will generate unwanted excitations which  to be accounted for during error correction.
Carefully tracking how the logical states transform under merging and splitting allows  to infer the  of quantum gates that can be implemented by surgery operations.
Typically  finds a larger  of gates and operations than what is available by pure braiding and measurement of bulk anyons.
Previous  has mapped out the s of gates and operations for several simple stabilizer codes~\cite{Horsman2012,Fowler2019,Landahl2014}. 
Such topological codes have an immediate generalization given by so-called quantum double models~\cite{Kitaev2003}. 
Gapped boundaries for these models and the  of allowed measurement operations have  been fully characterized~\cite{Kitaev2012,Cong2016,Beigi2011,Etingof2009}.
I propose to apply the general theory of gapped boundaries to develop a more complete theory of lattice surgery.


The first project in this direction is to generalize lattice surgery to a simple class of models in  spatial dimensions: non-chiral Abelian topological orders. 
In this very general ting, there is no  for the patches to have the same topological order, and  could study to what extent quantum states hosted in  topological order could be passed to another using, for instance, merging operations.
In these examples the gapped boundaries and possible measurements have a simple algebraic description~\cite{Beigi2011}.
Even so, there is no algebraic formalism in place to track and correct errors.
Once a large class of examples has been established, we can begin developing the algebraic formalism.
Specifically, I would like to parametrize the logical states and fault-tolerant operations in terms of the algebraic data of the bulk topological orders and the gapped boundaries. 
With all current examples, the  of gates  finds with lattice surgery are closely related to the  of gates found by braiding defects in a particular symmetry enriched topological order. 
Determining to what extent this conjecture is correct would provide a powerful handle on what gates and operations are accessible with lattice surgery.


In  spatial dimensions, lattice surgery has not been explored beyond the simplest examples~\cite{Vasmer2019}.
The problem is much more subtle, in large part due to the fact that gapped boundaries in  dimensions have not been fully characterized.
Ntheless, with the ongoing efforts in the fracton community significant progress has been made on gapped boundaries for  dimensional commuting projector models.
I will apply the techniques I developed in~\cite{Aasen2020defect} to determine possible gates and operations induced by lattice surgery of simple  dimensional topological orders.






{Concluding remarks}
{With the rapid development of quantum devices and materials the era of quantum computing is on the horizon; I am excited to  what we can accomplish. I have proposed a  program bringing together diverse areas of physics, from experimentally motivated materials science and device engineering, to abstract formalisms for more mathematical questions. My background and  program ideally situate me to complement the diverse group of world-class ers at the Stewart Blusson Quantum Matter Institute.
}